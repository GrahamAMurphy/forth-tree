\documentstyle[fancyheadings]{article}
\raggedbottom
\textwidth 6.5in
\oddsidemargin 0.0in
\evensidemargin 0.0in
\topmargin 0.0in
\textheight 9.0in

\pagestyle{fancy}
\newcommand{\memonum}{TSS-94-???}
\newcommand{\issuedate}{March 4, 1994}
\lhead{DRAFT}
\chead{}
\rhead{\memonum\\\ifnum\value{page}=1\issuedate\else Page \thepage\fi}
\lfoot{}
\cfoot{}
\rfoot{}
\addtolength{\headheight}{12pt}

\newcommand{\memoaddrlabel}[1]{\mbox{\bf #1}\hfil}
\newenvironment{memoaddress}{%
\begin{list}{}
	{
		\let\makelabel\memoaddrlabel
		\setlength{\labelwidth}{1in}
		\setlength{\leftmargin}{1in}
		\setlength{\labelsep}{0in}
		\setlength{\itemsep}{0in}
	}
}{%
\end{list}}

\newcommand{\doctype}{memo}			% memo/article/paper

% namelist environment: from LaTex for Scientists and Engineers.
\newcommand{\namelistlabel}[1]{\mbox{#1:}\hfil}
\newenvironment{namelist}[1]{%
\begin{list}{}
   {
      \let\makelabel\namelistlabel
      \settowidth{\labelwidth}{#1}
      \setlength{\leftmargin}{1.1\labelwidth}
   }
}{%
\end{list}}

\begin{document}
\begin{memoaddress}
	\item[To:] Distribution
	\item[Via:] R. M. Henshaw
	\item[From:] J. R. Hayes
	\item[Subject:] JH Forth Programmer's Manual
	\item[References:] Tracy, M., Anderson, A.,
	   \underline{Mastering Forth}, Brady Books, 1989.
	\item[] Hayes, J.R., ``Object-Oriented Programming for Small
	   Systems'', TES-91-182, November 1, 1991.
	\item[] Programming Languages - Forth, draft proposed
	   American National Standard for Information Systems,
	   X3J14 dpANS-6, June 30, 1993.
\end{memoaddress}

\newcommand{\fw}[1]{{\bf #1}}			% Forth words are in bold

\section{Introduction}
This \doctype\ describes the unique features of JH Forth.
JH Forth follows the ANS Forth standard.
Most of this manual is devoted to listing the optional parts of ANS
Forth that have been implemented in JH Forth and in
describing JH Forth's extensions to ANS Forth.

This manual is not a tutorial on Forth.  There are already a number of
introductory Forth textbooks.  ``Mastering Forth'' by Martin Tracy is a
good book for the novice.  Nor is this manual a tutorial on ANS Forth.
The code examples used in ``Mastering Forth'' are close enough to
ANS Forth that the book can be used with an ANS Forth system.

\section{ANS Forth}
Since JH Forth complies with the ANS Forth standard, the ANS
Forth standard document can be used as a reference manual for JH
Forth.  JH Forth implements the ANS Forth Core, Locals,
Search-Order, File-Access, Block, and Memory Allocation
word sets.

There is one area in which JH Forth does not conform to ANS Forth.
At APL, Forth source code is maintained as text files.
Handling traditional upper-case Forth source code with
the Unix utilities is awkward.  Consequently, lower-case is used
and JH Forth expects
lower-case source code instead of the upper-case source required
by ANS Forth.

In addition to words of the Core, Locals, Search-Order,
File-Access, Block, and Memory Allocation word sets,
JH Forth provides some useful words from other ANS Forth
word sets.  From the Core Extensions word set are:
\begin{center}
\fw{.(}\ \ \ \fw{.r}\ \ \ \fw{0$>$}\ \ \ 
\fw{:noname}\ \ \ \fw{$<>$}\ \ \ \fw{again}\ \ \ \fw{compile,} \\
\fw{convert}\ \ \
\fw{false}\ \ \ \fw{hex}\ \ \ \fw{nip}\ \ \ \fw{pad}\ \ \ 
\fw{parse}\ \ \ \fw{pick}\ \ \ \fw{roll}\ \ \ \fw{true}\ \ \ \fw{u.r}\ \ \ 
\fw{u$>$}\ \ \ \fw{within}\ \ \ \fw{\\}
\end{center}

From the Double-Number word set are:
\begin{center}
\fw{2variable}\ \ \ \fw{d+}\ \ \ \fw{d.}\ \ \ \fw{d.r}\ \ \ \fw{d$<$}\ \ \ 
\fw{dabs}\ \ \ \fw{dnegate}
\end{center}

From the Programming-Tools word set are:
\begin{center}
\fw{.s}\ \ \ \fw{?}\ \ \ \fw{bye}\ \ \ \fw{dump}\ \ \ \fw{forget}
\end{center}

From the String word set are:
\begin{center}
\fw{/string}\ \ \ \fw{cmove}\ \ \ \fw{cmove$>$}
\end{center}

\section{More Words}
JH Forth contains a number of words that have proven useful over
the years, but are not present in ANS Forth.
These words are documented
in an appended glossary.  Some of these words are also discussed in more
depth below.

\subsection{Cell-Size Independence}
ANS Forth provides two words, \fw{cell+} and \fw{cells}, for doing
cell-size independent address calculations.  JH Forth provides
a few extra operators in the same vein.  \fw{cell} returns the size
of a cell in address units, \fw{cell-} computes the address of the
preceding cell, and \fw{cell/} computes the number of cells
that could fit in a given number of address units.  The ANS Forth
Portability Guide is recommended reading for understanding how
to write cell-size independent code and portable Forth code in
general.

\subsection{Case Structure}
The case selection structure used in JH Forth is different from
the one most used by Forth programmers and merits some
explanation.  A simple example illustrates most of the capabilities of
the case structure:
\begin{verbatim}
   : .number       ( n --- )
      sel
         << 1    ==> ." one" >>
         << 2    ==> ." two" >>
         dup . ." is illegal"
      endsel ;
\end{verbatim}
The word \fw{.number} tests its argument and prints the string ``one''
if the value is one, ``two'' if it is two, and otherwise prints the
argument followed by the message ``is illegal''.  When the case
structure is entered at \fw{sel}, a case selector is on the top of the
data stack.  For each case, the selector is compared
non-destructively to its selection value between \fw{$<<$} and
\fw{==$>$}.  If the comparison returns true, the selector is destroyed
(\fw{drop}), the corresponding code between \fw{==$>$} and \fw{$>>$}
is executed, and execution continues after \fw{endsel}.  The use of
literals as a selection value is for illustration only.  Any Forth word
that returns a single value can be used.  If no match is found, an
optional fall-through clause is executed.  In the example, a
copy of the selector is made (\fw{dup}) for use by the code
in the clause.
Finally, \fw{endsel}, which is executed only if none of the cases
matched, destroys the case selector.

An extension to this structure is \fw{=$>$}, which allows more general
testing for a case.  Another version of \fw{.number} demonstrates its
use:
\begin{verbatim}
   : .number       ( n --- )
      sel
         << 0<           => ." negative" >>
         << 0=           => ." zero" >>
         << 1 6 within   => ." one through five" >>
      endsel ;
\end{verbatim}
\fw{$<<$} makes a copy of the case selector for testing.  The code
between \fw{$<<$} and \fw{=$>$} destructively tests the copy leaving true
or false on the stack.

\subsection{Context-independent Numbers}
The context-dependence of literal number interpretation can
detract from the readability of Forth programs.  The meaning
of ``10'', for example, depends on the current value of \fw{base};
if \fw{base} is hexadecimal, ``10'' is sixteen, but if \fw{base}
is decimal, ``10'' is ten.  Someone perusing a program has to
search back from the occurrence of a literal number and find
where \fw{base} is set to determine a literal's value.

In JH Forth, the words \fw{h } and \fw{d } are available
to momentarily override the current \fw{base} and force the interpretation
of the following number of be in hexadecimal or decimal respectively.
Example usage is as follows:
\begin{verbatim}
   h# 10 constant sixteen
   : foo       ( --- )
      sixteen d# 100 foobar	\ pass 16 and 100 to foobar
   ;
\end{verbatim}

\section{Source Code Libraries}
A number of utilities are provided in JH Forth in source code
form.
\begin{namelist}{xxxxxxxxxxxx}
	\item[blocks.fr] ANS Forth Block word set
	\item[env.fr] ANS environment query word: \fw{environment?}
        \item[locals.fr] ANS User-level local variable syntax
	\item[locals2.fr] Alternate user-level local variable syntax
	\item[modules.fr] Modules
        \item[onlyalso.fr] Vocabularies and \fw{only},
                \fw{also}, and \fw{previous}
        \item[struct.fr] Data structure defining words
        \item[eaker.fr] An alternative case structure (by Eaker)
        \item[foop3-1.fr] Forth Object-Oriented Programming (FOOP) system
\end{namelist}

JH Forth does not come with a block system installed;
an ANS Forth Block word set implementation can be found in ``blocks.fr''.
The ANS Forth environmental query word \fw{environment?} is implemented
in ``env.fr''.
The ANS Forth Locals word set provides a low-level local variable
construction utility.  A higher-level interface is defined in the Local
Extension word set and implemented in ``locals.fr''.  An alternative
interface is provided in ``locals2.fr''.  
The ANS Forth Search-Order word set
provides low-level control of word name scope and system search order.
A \fw{vocabulary} system with \fw{only}, \fw{also},
and \fw{previous}, a module mechanism, or an entire
object-oriented programming system
can be constructed using the Search Order word set.  JH Forth
sources for such systems are in the files ``onlyalso.fr'', ``modules.fr''
and ``foop3-1.fr'' respectively.
Usage of each utility is documented in its source code except for
the FOOP system, which is described in
``Object-oriented programming for Small Systems''.

These utilities can be loaded via \fw{used}, (e.g. \fw{" struct.fr" used}).
The arguments to \fw{used} are the address and length of a string
naming a file (as in \fw{included}.  \fw{used} searches for the file
using a list of directories.  The list of directories is obtained from
an environment variable named FPATH.  FPATH should contain directory
names separated by a space character; the list is examined left to right.
FPATH
should be set in your login startup script (e.g. .profile, .login,
autoexec.bat, etc.)
and contain the system-dependent location of the source code libraries.

\section{Invocation}
JH Forth is invoked from you shell's command libe by typing 'forth'.
This results in an interactive Forth session.  The session may be
terminated by typing your end-of-file character (typically \verb|^|D).  JH
Forth may also be run non-interactively by typing 'forth' followed
by the names of one or more files containing Forth source code.
The Forth system interprets the contents of each file without producing
any prompts.

\vspace{1in}
\hspace{4in}John R. Hayes

\begin{flushleft}
{\bf Distribution:} \\
BWBallard \\
HKCharles, Jr. \\
JRDettmer \\
MEFraeman \\
RMHenshaw \\
GDWagner \\
TES Files \\
Archives \\
\end{flushleft}
\end{document}
