\documentstyle[fancyheadings]{article}
\raggedbottom
\textwidth 6.5in
\oddsidemargin 0.0in
\evensidemargin 0.0in
\topmargin 0.0in
\textheight 9.0in

\pagestyle{fancy}
\newcommand{\memonum}{S1I-94-Draft}
\newcommand{\issuedate}{June 27, 1994}
\lhead{}
\chead{}
\rhead{\memonum\\\ifnum\value{page}=1\issuedate\else Page \thepage\fi}
\lfoot{}
\cfoot{}
\rfoot{}
\addtolength{\headheight}{12pt}

\newcommand{\memoaddrlabel}[1]{\mbox{\bf #1:}\hfil}
\newenvironment{memoaddress}{%
\begin{list}{}
	{
		\let\makelabel\memoaddrlabel
		\setlength{\labelwidth}{1in}
		\setlength{\leftmargin}{1in}
		\setlength{\labelsep}{0in}
		\setlength{\itemsep}{0in}
	}
}{%
\end{list}}

\newcommand{\doctype}{memo}			% memo/article/paper

\begin{document}
\begin{memoaddress}
	\item[To] Distribution
	\item[Via] R. M. Henshaw
	\item[From] J. R. Hayes
	\item[Subject] FRISC Forth on MPE's RTX Power Board
	\item[Reference] Hayes, J.R., ``FRISC Forth Programmer's
		Manual'', TES-92-45.
\end{memoaddress}

\newcommand{\fw}[1]{{\bf #1}}			% Forth words are in bold
\newcommand{\num}[1]{{\em #1}}			% italicize numbers

\section{Introduction}
This \doctype\ documents FRISC Forth as implemented on Microprocessor
Engineering Ltd.'s RTX Power Board.
The \doctype\ describes how to install FRISC Forth on the board,
how to use the system with a host computer, and
special features of the system.
FRISC Forth is described in general in ``FRISC Forth Programmer's
Manual''.

To install FRISC Forth on the board, the original EPROMs
must be replaced with FRISC Forth EPROMs.  The FRISC Forth EPROMs
are labeled ICJ and ICK; they should be inserted in the correspondingly
labeled sockets.

The FRISC Forth system uses an RS-232 interface wired as DCE.
On reset, the interface is programmed to use 9600 baud, 8 data bits,
no parity, and one stop bit.  The system processes some received
characters specially.  \verb|^|S and \verb|^|Q stop and start the
serial output.  Carriage returns are mapped to linefeeds.
\verb|^|C executes the \fw{reboot} command, reinitializing the system
while retaining the current dictionary.

\section{Interpreting/Compiling Software}
Forth source code can be entered interactively by typing in lines
of fewer than 128 characters.  The line may be edited:
\verb|^|H erases the previous character and \verb|^|U
discards the entire line.  The line is terminated by a carriage
return or linefeed.  The system returns an ``ok'' response
after each line (while in interpret state).

Forth source code can be downloaded from a host computer.
The host computer can be connected to the board's RS-232 line
and a terminal emulation program used for communication.
The command \fw{silent}
is typed to disable echoing and the ``ok'' response.  Files can
then be sent from the host computer as a stream of ASCII characters.
Each line must consist of 128 or fewer characters
followed by a carriage return or linefeed.
To terminate silent mode, \fw{abort} (or \fw{quit}) should be typed.

\section{Special RTX Features}
To make FRISC Forth appear uniform across different computer architectures,
the idiosyncratic personality traits of various processors have been
suppressed.  This makes it easier to write programs that will
run on other FRISC Forth systems (e.g. FRISC3, FRISC4, RTX2000,
RTX2010, etc.).  However, there may be situations where use of special
processor features may be advantageous.  For example, the RTX has
a loop instruction that is faster than the standard \fw{do}/\fw{loop}.

A RTX implementation of Forth standard \fw{do} ... \fw{loop} control
structure exacts a run-time toll of six clocks per iteration.  In
situations where this is excessive, the RTX's \fw{for} ... \fw{next}
control structure can be used instead. 
\fw{For}, given an argument \num{n},
executes the code between \fw{for} and \fw{next} \num{n} times;
if \num{n} is zero, the loop is not executed.  The loop index
counts down from \num{n-1} through 0 and is available by using \fw{r@}.
There is a one clock overhead per iteration.

An RTX instruction opcode may be given a name with the defining word
\fw{ucode}.  The instruction may be subsequently invoked or compiled
into a definition by using its name.  For example, the instruction
that reads the RTX configuration register may be defined:
\begin{verbatim}
   be03 ucode cr@
\end{verbatim}
When an instruction defined by \fw{ucode} is used in a subsequent
definition, the instruction's opcode is compiled in-line (instead of
being called).

\section{Memory}
FRISC Forth on the RTX can access 64k bytes of address space;
however a bank-switching mechanism allows access to additional
64 kbyte pages.  By default, all data fetches and stores are made to
page 0 (which also contains object code).  This default page
is called the ``near'' page; other pages are called ``far''
pages.

FRISC Forth has words to identify a far memory page,
and to switch back and forth between far and near pages.
\fw{set-page} records a page number, given as an
argument, to be used for future data accesses.
\fw{Far} puts the system in far mode
such that all data accesses occur in the page
identified to \fw{set-page}.  \fw{near} switches the system
back to near mode such that all data accesses occur
in the default page.

Note that using \fw{set-page} does not effect whether the
system is in far or near mode.  This means that the far page
of interest need be set only once.  A program can switch back
and forth from far and near space as often as necessary
using only \fw{far} and \fw{near}.  The behavior of \fw{far},
\fw{near}, and \fw{set-page} are summarized in the Appendix.
Some additional words for manipulating far memory (all implemented
in terms of the three primitives) are also described in the Appendix.

On reset, the Forth system copies itself from EPROM into RAM
(near space),
then maps the EPROM out of the address space.
The system uses memory starting at address 0; this memory region
grows towards higher memory addresses as words are added
to the dictionary or data space is \fw{allot}ted.  \fw{here} returns
the address of the next free memory location.

\section{Interrupts}
Interrupt routines may be written using the following syntax:
\begin{verbatim}
   n int: ... ;int
\end{verbatim}
\fw{int:} and \fw{;int} are analogous to \fw{:} (colon) and \fw{;}
(semicolon); both pairs of words demarcate Forth source code to be compiled.
\fw{Int:}, instead of associating the compiled code with a name in the
dictionary as \fw{:} (colon) does, associates the code with interrupt
\num{n}.  The number \num{n} identifies one of the interrupts
known to the RTX interrupt controller (see Table 1).
Default nop interrupt routines are preinstalled for all interrupts.

\begin{table*}
\caption{Interrupts}
\begin{center}
\begin{tabular}{|c|l|c|c|c|}
\hline
Number & Source &			IMR Mask &	Type &	Priority  \\
\hline \hline
16 &	No Interrupt &			na &		Edge &	na  \\
\hline
15 &	Nonmaskable (NMI) &		na &		Edge &	0 (highest)  \\
\hline
14 &	EI1 pin &			0xfffd &	Level &	1  \\
\hline
13 &	Parameter Stack Underflow &	0xfffb &	Level &	2  \\
\hline
12 &	Return Stack Underflow &	0xfff7 &	Level &	3  \\
\hline
11 &	Parameter Stack Overflow &	0xffef &	Level &	4  \\
\hline
10 &	Return Stack Overflow &		0xffdf &	Level &	5  \\
\hline
 9 &	EI2 pin &			0xffbf &	Level &	6  \\
\hline
 8 &	Timer 0 &			0xff7f &	Edge &	7  \\
\hline
 7 &	Timer 1 &			0xfeff &	Edge &	8  \\
\hline
 6 &	Timer 2 &			0xfdff &	Edge &	9  \\
\hline
 5 &	EI3 pin &			0xfbff &	Level &	10  \\
\hline
 4 &	EI4 pin &			0xf7ff &	Level &	11  \\
\hline
 3 &	EI5 pin &			0xefff &	Level &	12  \\
\hline
 2 &	Software Interrupt &		0xdfff &	Level &	13 (lowest)  \\
\hline
\end{tabular}
\end{center}
\end{table*}  


\fw{Int:} and \fw{;int} save and restore sufficient system state that
Forth source code may appear between \fw{int:} and \fw{;int}.
\fw{Int:} saves the MD and CR registers and
switches the system to use near address space (see
Memory above).
Further interrupts are disabled during the interrupt routine; interrupts
are re-enabled and registers restored by \fw{;int}.

Before an interrupt routine can execute, its corresponding interrupt
must be enabled.  The interrupt mask register (IMR) contains sixteen
bits (1 = disabled, 0 = enabled).  Table 1 shows the mask needed to
enable each of the interrupts.  To enable all of the interrupts
for an application, the appropriate masks from Table 1 should be
`and'ed together and stored into the IMR.
The words \fw{imr@} and \fw{imr!} are available for
reading and writing the IMR.

Interrupt latency, the time between the occurrence of an interrupt
and the start of the interrupt routine, is affected by several factors.
The RTX only responds to interrupts between instructions when
interrupts are enabled.  This reponse delay will be
the maximum of the longest time that interrupts are disabled and
the time of the longest RTX instruction.  The longest time that
interrupts are disabled, during an interrupt routine for example,
is application specific.  The longest RTX instructions are the
streamed instructions, which can take up to 65536 clocks to execute.
FRISC Forth on the RTX uses streaming during divide and shift
operations for example; the longest stream used is sixteen clocks.
Saving system
state creates additional delay.  The RTX adds one clock for an
interrupt response cycle and \fw{int:} adds TBD clocks
for saving registers.

\section{EEPROMs}
The Forth system has the ability to save the RAM memory image in EEPROM.
To use this feature, Xicor X28VC256 55ns EEPROMs can be inserted
in the sockets labeled ICH and ICI.  Software may be downloaded and
compiled on the MPE board, then saved in EEPROM; \fw{save} saves
the state of the system by copying RAM from address 0 through \fw{here}
into EEPROM.  The saved system may be restored with \fw{unsave}.

\vspace{1in}
\hspace{4in}John R. Hayes

\begin{flushleft}
{\bf Distribution:} \\
HKCharles, Jr. \\
JRDettmer \\
MEFraeman \\
RMHenshaw \\
GDWagner \\
RLWilliams \\
TES Files \\
Archives \\
\end{flushleft}
\end{document}
