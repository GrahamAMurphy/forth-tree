\begin{cgloss}{;int}{ -- }{ -- }

End compilation of interrupt routine.
\end{cgloss}
\begin{gloss}{blocks-in}{ src dst blocks -- src' dst'}

Copy given number of 16-byte blocks from source in far memory to
destination in near memory.  Returns addresses just past last byte copied.
Each block is streamed onto the stack, then streamed into near memory.
Consequently, interrupts will held off by up to 16 clocks.
\end{gloss}
\begin{gloss}{blocks-out}{ src dst blocks -- src' dst'}

Copy given number of 16-byte blocks from source in near memory to
destination in far memory.  Returns addresses just past last byte copied.
Each block is streamed onto the stack, then streamed into far memory.
Consequently, interrupts will held off by up to 16 clocks.
\end{gloss}
\begin{gloss}{bytes-in}{ src dst bytes -- }

Copy given number of bytes from source in far memory to destination
in near memory.
\end{gloss}
\begin{gloss}{bytes-out}{ src dst bytes -- }

Copy given number of bytes from source in near memory to destination
in far memory.
\end{gloss}
\begin{gloss}{far}{ -- }

Switch to far memory access mode.  All data fetches or stores will be
made with respect to the page most recently identified to \fw{set-page}
\end{gloss}
\begin{cgloss}{for}{ u -- }{ -- sys}

Compiling word indicates beginning of for/next loop.  Usage:
\begin{verbatim}
	: foo
	   ...
	   for  ...   next
	   ... ;
\end{verbatim}
The code between \fw{for} and \fw{next} is executed \type{u}
times.  The loop index counts down from \type{u-1} to 0
and may be read with \fw{r@}.
\end{cgloss}
\begin{gloss}{imr!}{ u -- }

Set Interrupt Mask Register (IMR) to \type{u}.
\end{gloss}
\begin{gloss}{imr@}{ -- u }

Return Interrupt Mask Register (IMR).
\end{gloss}
\begin{gloss}{int:}{ u -- }

Start compilation of interrupt routine for interrupt \type{u}.
When the interrupt routine executes, sufficient state will be saved
that most Forth code can be used in the routine.  Interrupts are
disabled.
\end{gloss}
\begin{gloss}{near}{ -- }

Switch to near memory access mode.  All data fetches or stores will be
made with respect to the default memory page (page 0).
\end{gloss}
\begin{cgloss}{next}{ -- }{ sys -- }

Compiling word indicates end of for/next loop.  See \fw{for}.
\end{cgloss}
\begin{gloss}{save}{ -- }

Save system state in EEPROM.
\end{gloss}
\begin{gloss}{set-page}{u -- }

Set page to be used on subequent far memory accesses to be \type{u}.
\end{gloss}
\begin{gloss}{ucode}{ u \verb|"|name\verb|"| -- }

Defining word associates RTX opcode \type{u} with dictionary entry
\type{name}.  When \type{name} is used in a subsequent definition,
the opcode \type{u} is compiled in-line.
\end{gloss}
\begin{gloss}{unsave}{ -- }

Restore system state from EEPROM.
\end{gloss}
