\documentstyle[fancyheadings]{article}
\raggedbottom
\textwidth 6.5in
\oddsidemargin 0.0in
\evensidemargin 0.0in
\topmargin 0.0in
\textheight 9.0in

\pagestyle{fancy}
\newcommand{\memonum}{TES-92-45}
\newcommand{\issuedate}{February 27, 1992}
\lhead{}
\chead{}
\rhead{\memonum\\\ifnum\value{page}=1\issuedate\else Page \thepage\fi}
\lfoot{}
\cfoot{}
\rfoot{}
\addtolength{\headheight}{12pt}

\newcommand{\memoaddrlabel}[1]{\mbox{\bf #1}\hfil}
\newenvironment{memoaddress}{%
\begin{list}{}
	{
		\let\makelabel\memoaddrlabel
		\setlength{\labelwidth}{1in}
		\setlength{\leftmargin}{1in}
		\setlength{\labelsep}{0in}
		\setlength{\itemsep}{0in}
	}
}{%
\end{list}}

\newcommand{\doctype}{memo}			% memo/article/paper

% namelist environment: from LaTex for Scientists and Engineers.
\newcommand{\namelistlabel}[1]{\mbox{#1:}\hfil}
\newenvironment{namelist}[1]{%
\begin{list}{}
   {
      \let\makelabel\namelistlabel
      \settowidth{\labelwidth}{#1}
      \setlength{\leftmargin}{1.1\labelwidth}
   }
}{%
\end{list}}

\begin{document}
\begin{memoaddress}
	\item[To:] Distribution
	\item[Via:] R. M. Henshaw
	\item[From:] J. R. Hayes
	\item[Subject:] FRISC Forth Programmer's Manual
	\item[References:] Tracy, M., Anderson, A.,
	   \underline{Mastering Forth}, Brady Books, 1989.
	\item[] Hayes, J.R., ``Object-Oriented Programming for Small
	   Systems'', TES-91-182, November 1, 1991.
	\item[] Programming Languages - Forth, draft proposed
	   American National Standard for Information Systems,
	   X3J14 dpANS-2, August 3, 1991.
\end{memoaddress}

\newcommand{\fw}[1]{{\bf #1}}			% Forth words are in bold

\section{Introduction}
This \doctype\ describes the unique features of Forth implemented on
Forth Reduced Instruction Set Computer
(FRISC) microprocessors.  FRISC Forth was designed to be ``ordinary''.
There are many unusual features in the processor that are hidden from
the programmer.  This allows a Forth programmer who is used to working
with Forth on conventional computers to take advantage of that
experience when programming the FRISC.  Ideally, the only way he will be
able to tell that he is using the FRISC instead of a conventional machine
is that the FRISC will run much faster.  The low level features of the
chip are available to the intrepid via the assembler.

FRISC Forth follows the ANS Forth standard.
Most of this manual is devoted to listing the optional parts of ANS
Forth that have been implemented in FRISC Forth and in
describing FRISC Forth's extensions to ANS Forth.

This manual is not a tutorial on Forth.  There are already a number of
introductory Forth textbooks.  ``Mastering Forth'' by Martin Tracy is a
good book for the novice.  Nor is this manual a tutorial on ANS Forth.
The code examples used in ``Mastering Forth'' are close enough to
ANS Forth that the book can be used with an ANS Forth system.

\section{ANS Forth}
Since FRISC Forth complies with the ANS Forth standard, the ANS
Forth standard document can be used as a reference manual for FRISC
Forth.  FRISC Forth implements the ANS Forth Core, Locals,
and Search-Order word sets.

There is one area in which FRISC Forth does not conform to ANS Forth.
At APL, Forth source code is maintained as text files on Unix
systems.  Handling traditional upper-case Forth source code with
the Unix utilities is awkward.  Consequently, lower-case is used
and FRISC Forth expects
lower-case source code instead of the upper-case source required
by ANS Forth.

In addition to words of the Core, Locals, and Search-Order word sets,
FRISC Forth provides some useful words from other ANS Forth
word sets.  From the Core Extensions word set are:
\begin{center}
\fw{.(}\ \ \ \fw{.r}\ \ \ \fw{0$>$}\ \ \ 
\fw{:noname}\ \ \ \fw{$<>$}\ \ \ \fw{again}\ \ \ \fw{compile,} \\
\fw{convert}\ \ \
\fw{false}\ \ \ \fw{hex}\ \ \ \fw{j}\ \ \ \fw{nip}\ \ \ \fw{pad}\ \ \ 
\fw{parse}\ \ \ \fw{true}\ \ \ \fw{u.r}\ \ \ \fw{u$>$}\ \ \
\fw{within}\ \ \ \fw{\\}
\end{center}

From the Double-Number word set are:
\begin{center}
\fw{2variable}\ \ \ \fw{d+}\ \ \ \fw{d.}\ \ \ \fw{d.r}\ \ \ \fw{d$<$}\ \ \ 
\fw{dabs}\ \ \ \fw{dnegate}
\end{center}

From the Programming-Tools Extensions word set are:
\begin{center}
\fw{?}\ \ \ \fw{dump}\ \ \ \fw{forget}
\end{center}

From the String word set are:
\begin{center}
\fw{cmove}\ \ \ \fw{cmove$>$}
\end{center}

\section{More Words}
FRISC Forth contains a number of words that have proven useful over
the years, but are not present in ANS Forth.
These words are documented
in an appended glossary.  Some of these words are also discussed in more
depth below.

\subsection{Cell-Size Independence}
ANS Forth provides two words, \fw{cell+} and \fw{cells}, for doing
cell-size independent address calculations.  FRISC Forth provides
a few extra operators in the same vein.  \fw{cell} returns the size
of a cell in address units, \fw{cell-} computes the address of the
preceding cell, and \fw{cell/} computes the number of cells
that could fit in a given number of address units.  The ANS Forth
Portability Guide is recommended reading for understanding how
to write cell-size independent code and portable Forth code in
general.

\subsection{Case Structure}
The case selection structure used in FRISC Forth is different from
the one most used by Forth programmers and merits some
explanation.  A simple example illustrates most of the capabilities of
the case structure:
\begin{verbatim}
   : .number       ( n --- )
      sel
         << 1    ==> ." one" >>
         << 2    ==> ." two" >>
         dup . ." is illegal"
      endsel ;
\end{verbatim}
The word \fw{.number} tests its argument and prints the string ``one''
if the value is one, ``two'' if it is two, and otherwise prints the
argument followed by the message ``is illegal''.  When the case
structure is entered at \fw{sel}, a case selector is on the top of the
data stack.  For each case, the selector is compared
non-destructively to its selection value between \fw{$<<$} and
\fw{==$>$}.  If the comparison returns true, the selector is destroyed
(\fw{drop}), the corresponding code between \fw{==$>$} and \fw{$>>$}
is executed, and execution continues after \fw{endsel}.  The use of
literals as a selection value is for illustration only.  Any Forth word
that returns a single value can be used.  If no match is found, an
optional fall-through clause is executed.  In the example, a
copy of the selector is made (\fw{dup}) for use by the code
in the clause.
Finally, \fw{endsel}, which is executed only if none of the cases
matched, destroys the case selector.

An extension to this structure is \fw{=$>$}, which allows more general
testing for a case.  Another version of \fw{.number} demonstrates its
use:
\begin{verbatim}
   : .number       ( n --- )
      sel
         << 0<           => ." negative" >>
         << 0=           => ." zero" >>
         << 1 6 within   => ." one through five" >>
      endsel ;
\end{verbatim}
\fw{$<<$} makes a copy of the case selector for testing.  The code
between \fw{$<<$} and \fw{=$>$} destructively tests the copy leaving true
or false on the stack.

\subsection{In-line Code Expansion}
When Forth's kernel primitives are used within a colon definition, the
FRISC Forth compiler copies the object code of the primitive into the
colon definition.  One cell at a time is copied until a
return-from-subroutine instruction is found.  The return is not copied.
Definitions that are expanded in-line should end with a return so that
they can also be run via vectored execution.

New operators defined by the programmer can be designated \fw{inline}.
\fw{Inline} is used, like \fw{immediate}, after the definition.
\fw{Inline} sets an in-line bit in the name field.  Operators that have
branches or are defined by \fw{create} can not be brought in-line.
\fw{Constant}s and \fw{variable}s all have their in-line bits set
automatically by the compiler.

\section{Source Code Libraries}
A number of utilities are provided in FRISC Forth in source code
form.
\begin{namelist}{xxxxxxxxxxxx}
        \item[asm.fr] Assembler
        \item[locals.fr] User-level local variable syntax
        \item[onlyalso.fr] Vocabularies and \fw{only},
                \fw{also}, and \fw{previous}
        \item[struct.fr] Data structure defining words
        \item[eaker.fr] An alternative case structure (by Eaker)
        \item[foop.fr] Forth Object-Oriented Programming (FOOP) system
\end{namelist}

The ANS Forth Locals word set provides a low-level local variable
construction utility.  A higher-level interface can be defined by the
user.  Such an interface can be found in the file ``locals.fr''.
In a similar spirit, the ANS Forth
Search-Order word set
provides low-level control of word name scope and system search order.
A traditional \fw{vocabulary} system with \fw{only}, \fw{also},
and \fw{previous} or an entire object-oriented programming system
can be constructed using the Search Order word set.  FRISC Forth
sources for such systems are in the files ``onlyalso.fr'' and
``foop.fr''.
Usage of each utility is documented in its source code except for
the FOOP system, which is described in
``Object-oriented programming for Small Systems''.

\vspace{1in}
\hspace{4in}John R. Hayes

\begin{flushleft}
{\bf Distribution:} \\
HKCharles, Jr. \\
JRDettmer \\
MEFraeman \\
RMHenshaw \\
GDWagner \\
RLWilliams \\
TES Files \\
Archives \\
\end{flushleft}
\end{document}
