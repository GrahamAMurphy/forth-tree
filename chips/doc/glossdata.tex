\begin{gloss}{"}{\verb|"|ccc$<$\verb|"|$>$\verb|"| -- c-addr u}

Used in the form:
\begin{verbatim}
   " ccc"
\end{verbatim}
Returns the address and length of the given string.  The string is stored
in a transient area.
\end{gloss}
\begin{cgloss}{""}{ -- c-addr u}{\verb|"|ccc$<$\verb|"|$>$\verb|"| -- }

Compiling word used in the form:
\begin{verbatim}
   : foo ... "" ccc" ... ;
\end{verbatim}
Compiles a string from the input stream into the current definition.
At run time, the address and length of the string are returned.
\end{cgloss}
\begin{gloss}{-rot}{x1 x2 x3 -- x3 x1 x2}

The top three stack entries are rotated, moving the top entry to the
bottom.  \fw{-rot} is the converse of \fw{rot}.
\end{gloss}
\begin{gloss}{2pick}{x1 x2 x3 -- x1 x2 x3 x1}

Shorthand for \fw{2 pick}.
\end{gloss}
\begin{gloss}{3pick}{x1 x2 x3 x4 -- x1 x2 x3 x4 x1}

Shorthand for \fw{3 pick}.
\end{gloss}
\begin{cgloss}{$<<$}{ x -- x x}{sys1 -- sys2}

Compiling word to signal the beginning of a case inside the select case
control structure.  See \fw{sel} for an example of how to use the case words.
\end{cgloss}
\begin{cgloss}{==$>$}{x1 x1 x2 -- x1 $|$ }{sys1 -- sys2}

Compiling word separates a case structure equality test from the corresponding
case action.  See \fw{sel} for an example of how to use the case words.
\end{cgloss}
\begin{cgloss}{=$>$}{x1 flag -- x1 $|$ }{sys1 -- sys2}

Compiling word separates a case structure test from the corresponding case
action.  See \fw{sel} for an example of how to use the case words.
\end{cgloss}
\begin{cgloss}{$>>$}{ -- }{sys1 -- sys2}

Compiling word ends a case inside a case select control structure.  See
\fw{sel} for an example of how to use the case words.
\end{cgloss}
\begin{gloss}{cell}{ -- u}

Constant that returns the size of a cell in address units.
\end{gloss}
\begin{gloss}{cell-}{a-addr1 -- a-addr2}

Subtract the size of a cell, specified in address units, from
\type{a-addr1} giving \type{a-addr2}.
\end{gloss}
\begin{gloss}{cell/}{n1 -- n2}

\type{n2} is the size, in cells (rounded towards zero), of
\type{n1} address units.
\end{gloss}
\begin{gloss}{d.lz}{d n -- }

Print the double precision number \type{d} right-justified in a field of
\type{n} characters with leading zeros prepended.
\end{gloss}
\begin{gloss}{disable}{ -- }

Disable interrupts.
\end{gloss}
\begin{gloss}{enable}{ -- }
Enable interrupts.
\end{gloss}
\begin{cgloss}{endsel}{x -- }{sys1 -- sys2}

Compiling word ends a case control structure.  See \fw{sel} for an example
of how to use the case words.
\end{cgloss}
\begin{gloss}{inline}{ -- }

Marks the most recently created dictionary entry as a word that should be
expanded inline (instead of being called) when used in a colon definition.
NOTE: should not be used on words that contain branches
or word defined via \fw{create}.
\end{gloss}
\begin{gloss}{m/mmod}{ud1 u1 -- u2 ud2}

Unsigned mixed-mode division.  Dividend and quotient are two cells big.
\end{gloss}
\begin{gloss}{nop}{ -- }

Do nothing.
\end{gloss}
\begin{gloss}{not}{x1 -- x2}

\type{x2} is the one's complement of \type{x1}.  Equivalent to \fw{invert}.
\end{gloss}
\begin{gloss}{number}{c-addr -- n true $<$or$>$ false}

\type{c-addr} points to a counted string.  \fw{number}
attempts to convert this string
to a number using the current base.  The converted number \type{n}
and a true flag
are returned if successful.  Otherwise, a false is returned.  For the
conversion to be successful, there must be a blank at the end
of the string.
\end{gloss}
\begin{gloss}{rdrop}{ -- }

Drop one item off the return stack.
\end{gloss}
\begin{gloss}{reboot}{ ... -- }

Reinitialize system.  All processes are killed and the Forth
interpretation process is restarted.  Dictionary entries are
retained.
\end{gloss}
\begin{gloss}{rotate}{x1 n -- x2}

Rotate \type{x1} by \type{n} bits.  If \type{n} is greater than zero,
\type{x1} is rotated left.  If
\type{n} is less than zero, \type{x1} is rotated right.  If \type{n}
is zero, nothing happens.
\end{gloss}
\begin{cgloss}{sel}{ -- }{sys1 -- sys2}

Compiling case structure word used in the form:
\begin{verbatim}
   : foo
      ...
      <selector> sel
         <<    1      ==> ... >>
         <<    2      ==> ... >>
         <<    5      ==> ... >>
      endsel
	  ... ;
\end{verbatim}
The constants 1, 2, and 5 are just shown as an example.  Any word that
leaves
one item on the stack can be used in the select field.  The action code 
symbolized by ..., can be any thing including another case structure.  The
selector is no longer on the stack when the action code begins execution.
If none of the words in the select
fields match the selector, the selector is dropped by \fw{endsel}.
\end{cgloss}
\begin{gloss}{touch}{a-addr -- }

Touch an address for side-effect.  Equivalent to \fw{@ drop}.
\end{gloss}
\begin{gloss}{u.lz}{u n -- }

\type{u} is displayed as an unsigned number right-justified in a field
of \type{n} characters with leading zeros.
\end{gloss}
\begin{gloss}{u?}{a-addr -- }

Display the contents of \type{a-addr} as an unsigned number
in free-field format.
\end{gloss}
\begin{gloss}{um*m}{ud1 u -- ud2}

\type{ud2} is the double precision product of the unsigned multiplication
of \type{ud1} and \type{u}.
\end{gloss}
\begin{gloss}{umax}{u1 u2 -- u3}

\type{u3} is the greater of \type{u1} and \type{u2}
according to the operation of \fw{u$>$}.
\end{gloss}
\begin{gloss}{umin}{u1 u2 -- u3}

\type{u3} is the lesser of \type{u1} and \type{u2}
according to the operation of \fw{u$<$}.
\end{gloss}
